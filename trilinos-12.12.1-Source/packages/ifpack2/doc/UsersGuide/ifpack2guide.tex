%
% $Id: SANDExampleReportNotstrict.tex,v 1.26 2009-05-01 20:59:19 rolf Exp $
%
% This is an example LaTeX file which uses the SANDreport class file.
% It shows how a SAND report should be formatted, what sections and
% elements it should contain, and how to use the SANDreport class.
% It uses the LaTeX report class, but not the strict option.
%
% Get the latest version of the class file and more at
%    http://www.cs.sandia.gov/~rolf/SANDreport
%
% This file and the SANDreport.cls file are based on information
% contained in "Guide to Preparing {SAND} Reports", Sand98-0730, edited
% by Tamara K. Locke, and the newer "Guide to Preparing SAND Reports and
% Other Communication Products", SAND2002-2068P.
% Please send corrections and suggestions for improvements to
% Rolf Riesen, Org. 9223, MS 1110, rolf@cs.sandia.gov
%
\documentclass[pdf,12pt,report]{SANDreport}
\usepackage{algpseudocode}
\usepackage{amsthm}
\usepackage{booktabs}
\usepackage{calc}
\usepackage{eso-pic}
\usepackage{fancyhdr}
\usepackage{ifthen}
\usepackage{indentfirst}
\usepackage{geometry}
\usepackage{graphicx}
\usepackage[colorlinks, bookmarksopen, %pagebackref=true, backref=page,
             linkcolor={blue},
             anchorcolor={black},
             citecolor={blue},
             filecolor={magenta},
             menucolor={blue},
             pagecolor={red},
             plainpages=false,pdfpagelabels,
             pdfauthor={Andrey Prokopenko, Chris Siefert, Jonathan J. Hu, Mark
             Hoemmen, Alicia Klinvex},
             pdftitle={Ifpack2 User's Guide},
             pdfkeywords={Ifpack2,preconditioners,guide,user},
             urlcolor={blue}]{hyperref}
\usepackage{listings}
\usepackage{mathptmx}	% Use the Postscript Times font
\usepackage{multirow}
\usepackage{pifont}
\usepackage[FIGBOTCAP,normal,bf,tight]{subfigure}
\usepackage{tabularx}
\usepackage{verbatim}
\usepackage{xspace}
\usepackage{flowchart} % also loads tikz
\usepackage{algorithm}
\usetikzlibrary{arrows}

%\usepackage{draftwatermark}
%\SetWatermarkScale{.5}

\algrenewcommand{\algorithmiccomment}[1]{\hskip3em // #1}

%%%%%%%%%%%%%%%%%%%%%%%%%%%%%%%%%%%%%%%%%%%%%%%%%%%%%%%%%%%%%%%%%%%%%%%%%%%%%%%%%%%%%%%%%%%%%%%%%%%%%%%%%%%%%%%%%%%%%
% Want larger todonotes on margins?
% First, use package showframes to show the frames
% Then, adjust the geometry
% NOTE: this must be removed in the final version
% \usepackage{showframe}
% \setlength{\marginparwidth}{3.5cm}

% Add disable to todonotes options to disable all TODO notes without removing them
% \usepackage[colorinlistoftodos,prependcaption,textsize=small]{todonotes}

% \usepackage{xargs}
% \usepackage{soul}
% \newcommandx{\fix}     [3][1=]{\todo[linecolor=red,backgroundcolor=red!25,bordercolor=red,#1]{\textbf{#2: }#3}}
% \newcommandx{\unsure}  [3][1=]{\todo[linecolor=green,backgroundcolor=green!25,bordercolor=green,#1]{\textbf{#2: }#3}}
% \newcommandx{\improve} [3][1=]{\todo[linecolor=blue,backgroundcolor=blue!25,bordercolor=blue,#1]{\textbf{#2: }#3}}
% \newcommandx{\info}    [3][1=]{\todo[linecolor=gray,backgroundcolor=gray!25,bordercolor=gray,#1]{\textbf{#2: }#3}}
% \newcommandx{\fixhl}   [2]    {\texthl{#1}\fix{#2}}
%%%%%%%%%%%%%%%%%%%%%%%%%%%%%%%%%%%%%%%%%%%%%%%%%%%%%%%%%%%%%%%%%%%%%%%%%%%%%%%%%%%%%%%%%%%%%%%%%%%%%%%%%%%%%%%%%%%%%


% If you want to relax some of the SAND98-0730 requirements, use the "relax"
% option. It adds spaces and boldface in the table of contents, and does not
% force the page layout sizes.
% e.g. \documentclass[relax,12pt]{SANDreport}
%
% You can also use the "strict" option, which applies even more of the
% SAND98-0730 guidelines. It gets rid of section numbers which are often
% useful; e.g. \documentclass[strict]{SANDreport}



% ---------------------------------------------------------------------------- %
%
% Set the title, author, and date
%
\title{Ifpack2 User's Guide 1.0 \\
(Trilinos version 12.6)}

\author{
  Andrey Prokopenko \\
  Scalable Algorithms \\
  Sandia National Laboratories\\
  Mailstop 1318 \\
  P.O.~Box 5800 \\
  Albuquerque, NM 87185-1318\\
  aprokop@sandia.gov\\
  \and
  Christopher Siefert \\
  Computational Math \& Algorithms \\
  Sandia National Laboratories\\
  Mailstop 1318 \\
  P.O.~Box 5800 \\
  Albuquerque, NM 87185-1318 \\
  \and
  Jonathan J. Hu \\
  Scalable Algorithms \\
  Sandia National Laboratories\\
  Mailstop 9159 \\
  P.O.~Box 0969 \\
  Livermore, CA 94551-0969\\
  jhu@sandia.gov \\
  \and
  Mark Hoemmen \\
  Scalable Algorithms \\
  Sandia National Laboratories\\
  Mailstop 1320 \\
  P.O.~Box 5800 \\
  Albuquerque, NM 87185-1318\\
  mhoemme@sandia.gov\\
  \and
  Alicia Klinvex \\
  Scalable Algorithms \\
  Sandia National Laboratories\\
  Mailstop 1320 \\
  P.O.~Box 5800 \\
  Albuquerque, NM 87185-1318\\
  amklinv@sandia.gov\\
}

% There is a "Printed" date on the title page of a SAND report, so
% the generic \date should generally be empty.
\date{}

\def\optionbox#1#2{\noindent$\hphantom{ii}${\parbox[t]{1.5in}{\it
#1}}{\parbox[t]{4.8in}{#2}} \\[1.1em]}

\def\choicebox#1#2{\noindent$\hphantom{th}$\parbox[t]{2.5in}{\sf
#1}\qquad\parbox[t]{3.55in}{#2}\\[0.8em]}

\def\structbox#1#2{\noindent$\hphantom{hix}${\parbox[t]{2.10in}{\it
#1}}{\parbox[t]{3.9in}{#2}} \\[.02cm]}

\def\protobox#1{\vspace{2em}{\flushleft{\bf Prototype}
\hrulefill}\flushleft{\fbox{\parbox[t]{6in}{\vspace{1em}{\sf
#1}\vspace{1em}}}}}


%\setlength{\oddsidemargin} {0.1\oddsidemargin}
%\setlength{\evensidemargin}{0.5\evensidemargin}
%\setlength{\topmargin}     {0.0\topmargin}
%\setlength{\textheight}    {1.16\textheight}
%\setlength{\textwidth}     {1.35\textwidth}

\newcommand{\amesos}       {\textsc{Amesos}\xspace}
\newcommand{\amesostwo}    {\textsc{Amesos2}\xspace}
\newcommand{\anasazi}      {\textsc{Anasazi}\xspace}
\newcommand{\aztecoo}      {\textsc{AztecOO}\xspace}
\newcommand{\belos}        {\textsc{Belos}\xspace}
\newcommand{\epetra}       {\textsc{Epetra}\xspace}
\newcommand{\epetraext}    {\textsc{EpetraExt}\xspace}
\newcommand{\galeri}       {\textsc{Galeri}\xspace}
\newcommand{\ifpack}       {\textsc{Ifpack}\xspace}
\newcommand{\ifpacktwo}    {\textsc{Ifpack2}\xspace}
\newcommand{\kokkosclassic}{\textsc{KokkosClassic}\xspace}
\newcommand{\loca}         {\textsc{Loca}\xspace}
\newcommand{\ml}           {\textsc{ML}\xspace}
\newcommand{\muelu}        {\textsc{MueLu}\xspace}
\newcommand{\nox}          {\textsc{NOX}\xspace}
\newcommand{\stratimikos}  {\textsc{Stratimikos}\xspace}
\newcommand{\teuchos}      {\textsc{Teuchos}\xspace}
\newcommand{\tpetra}       {\textsc{Tpetra}\xspace}
\newcommand{\tpetrakernels}{\textsc{TpetraKernels}\xspace}
\newcommand{\trilinos}     {\textsc{Trilinos}\xspace}
\newcommand{\xpetra}       {\textsc{Xpetra}\xspace}
\newcommand{\zoltan}       {\textsc{Zoltan}\xspace}
\newcommand{\zoltantwo}    {\textsc{Zoltan2}\xspace}

\newcommand{\klu}          {\textsc{Klu}\xspace}
\newcommand{\metis}        {\textsc{Metis}\xspace}
\newcommand{\mumps}        {\textsc{Mumps}\xspace}
\newcommand{\umfpack}      {\textsc{Umfpack}\xspace}
\newcommand{\superlu}      {\textsc{SuperLU}\xspace}
\newcommand{\superludist}  {\textsc{SuperLU\_dist}\xspace}
\newcommand{\parmetis}     {\textsc{ParMetis}\xspace}
\newcommand{\paraview}     {\textsc{ParaView}\xspace}

\newcommand{\parameterlist}{\texttt{Teuchos::ParameterList}\xspace}

\newcommand \trilinosWeb   {trilinos.sandia.gov\xspace}

\newcommand \true {\texttt{true}}
\newcommand \false{\texttt{false}}

\newcommand{\be}  {\begin{enumerate}}
\newcommand{\ee}  {\end{enumerate}}

\newcommand{\comm}[2]{\bigskip
                      \begin{tabular}{|p{4.5in}|}\hline
                      \multicolumn{1}{|c|}{{\bf Comment by #1}}\\ \hline
                      #2\\ \hline
                      \end{tabular}\\
                      \bigskip
                     }

% specify a Teuchos::Parameter.
%  #1 = parameter name
%  #2 = type
%  #3 = description
%  #4 = default value
% example:  \ccc{fact: ilut level-of-fill}{int}{1}{Sets the level-of-fill for ILUT.}
\newcommand{\ccc}[4]{\choicebox{\tt "#1"}{[{\tt #2}] #4 {\bf Default:~}#3.}}
\newcommand{\cccc}[2]{\choicebox{\tt "#1"}{#2}}


\newtheorem*{mycomment}{\ding{42}}
\newtheoremstyle{plain}
  {\topsep}   % ABOVESPACE
  {\topsep}   % BELOWSPACE
  {\normalfont}  % BODYFONT
  {0pt}       % INDENT (empty value is the same as 0pt)
  {\bfseries} % HEADFONT
  {}         % HEADPUNCT
  {5pt plus 1pt minus 1pt} % HEADSPACE
  {}          % CUSTOM-HEAD-SPEC

% further declarations and additional commands
\definecolor{hellgelb}{rgb}{1,1,0.8}   % background color for C++ listings
\definecolor{darkgreen}{rgb}{0.0, 0.2, 0.13}
%\definecolor{hellrot}{HTML}{FFA4C2}    % background color for xml files

% settings for listings package
\lstset{
  backgroundcolor=\color{hellgelb},
  basicstyle=\ttfamily\small,
  breakautoindent=true,
  breaklines=true,
  captionpos=b,
  columns=flexible,
  commentstyle=\color{darkgreen},
  extendedchars=true,
  float=hbp,
  frame=single,
  identifierstyle=\color{black},
  keywordstyle=\color{blue},
  numbers=none,
  numberstyle=\tiny,
  showspaces=false,
  showstringspaces=false,
  stringstyle=\color{purple},
  tabsize=2,
}


% ---------------------------------------------------------------------------- %
% Set some things we need for SAND reports. These are mandatory
%
\SANDnum{SAND2016-5338}
\SANDprintDate{June 2016}
\SANDauthor{Andrey Prokopenko, Christopher M. Siefert, Jonathan J. Hu, \\Mark
Hoemmen, Alicia Klinvex}


% ---------------------------------------------------------------------------- %
% Include the markings required for your SAND report. The default is "Unlimited
% Release". You may have to edit the file included here, or create your own
% (see the examples provided).
%
% \include{MarkUR} % Not needed for unlimted release reports


% ---------------------------------------------------------------------------- %
% The following definition does not have a default value and will not
% print anything, if not defined
%
%\SANDsupersed{SAND1901-0001}{January 1901}
%\input{MarkOUO}


% ---------------------------------------------------------------------------- %
%
% Start the document
%
\begin{document}

    \maketitle

    % ------------------------------------------------------------------------ %
    % An Abstract is required for SAND reports
    %
    \begin{abstract}
	%This is the definitive user guide for the \muelu{} library in Trilinos version XX.YY.
%\muelu{} is a C++ multigrid framework that can work with either the \epetra or \tpetra linear
%algebra libraries.
%\muelu{} provides a variety of aggregation-based multigrid algorithms,
%including smoothed aggregation algebraic multigrid (AMG), Petrov-Galerkin AMG, and AMG for
%Maxwell's equations, as well as many different types of smoothers.
%\muelu{} is templated on the index, scalar, and compute node types.
%Thus it is possible to use \muelu{} on problems with scalar types other than double, on very
%large problems, and to exploit node-level parallelism.

This is the official user guide for \muelu{} multigrid library in Trilinos
version~\input{version}. This guide provides an overview of \muelu, its capabilities, and
instructions for new users who want to start using \muelu{} with a minimum of
effort. Detailed information is given on how to drive \muelu{} through its XML
interface. Links to more advanced use cases are given. This guide gives
information on how to achieve good parallel performance, as well as how to
introduce new algorithms. Finally, readers will find a comprehensive listing of
available \muelu{} options.  {\em Any options not documented in this manual
should be considered strictly experimental.}

    \end{abstract}


    % ------------------------------------------------------------------------ %
    % An Acknowledgement section is optional but important, if someone made
    % contributions or helped beyond the normal part of a work assignment.
    % Use \section* since we don't want it in the table of context
    %
    \clearpage
    \chapter*{Acknowledgment}
	Many people have helped develop \ifpacktwo{}, and we would like to acknowledge
their contributions here: Ross Bartlett, Tom Benson, Erik Boman, Joshua Booth,
Julian Cortial, Kevin Deweese, Jeremie Gaidamour, Paul Lin, Travis Fisher, Sarah
Osborn, Eric Phipps, and Paul Tsuji. Finally, Alan Williams did the original
port from \ifpack{} and was the original lead developer of \ifpacktwo{}.


    % ------------------------------------------------------------------------ %
    % The table of contents and list of figures and tables
    % Comment out \listoffigures and \listoftables if there are no
    % figures or tables. Make sure this starts on an odd numbered page
    %
    \cleardoublepage		% TOC needs to start on an odd page
    \tableofcontents
    \listoffigures
    \listoftables


    % ---------------------------------------------------------------------- %
    % An optional preface or Foreword
    %\clearpage
    %\chapter*{Preface}
    %\addcontentsline{toc}{chapter}{Preface}
	%\input{CommonPreface}


    % ---------------------------------------------------------------------- %
    % An optional executive summary
    %\clearpage
    %\chapter*{Summary}
    %\addcontentsline{toc}{chapter}{Summary}
	%\input{CommonSummary}


    % ---------------------------------------------------------------------- %
    % An optional glossary. We don't want it to be numbered
    %\clearpage
    %\chapter*{Nomenclature}
    %\addcontentsline{toc}{chapter}{Nomenclature}
    %\begin{description}
	%\item[dry spell]
	%    using a dry erase marker to spell words
	%\item[dry wall]
	%    the writing on the wall
	%\item[dry humor]
	%    when people just do not understand
	%\item[DRY]
	%    Don't Repeat Yourself
    %\end{description}


    % ---------------------------------------------------------------------- %
    % This is where the body of the report begins; usually with an Introduction
    %
    \SANDmain		% Start the main part of the report

    %-----------------------------%
    % \chapter{Introduction}\label{sec:introduction}
    %-----------------------------%
    % \chapter{INTRODUCTION} 
The Software Utilities Package for the Engineering
Sciences (SUPES) is a collection of subprograms which perform frequently
used non-numerical services for the engineering applications programmer.
The
three functional categories of SUPES are: (1) input command parsing, (2) dynamic
memory management, and (3) system dependent utilities.  The subprograms in
categories one and two are written in standard FORTRAN-77~\cite{ansi},
while the subprograms
in category three are written in the C programming language.
Thus providing a standardized FORTRAN interface to
several system dependent features across a variety of hardware configurations
while using a single set of source files.
This feature can be viewed as a maintenance aid from
several perspectives.
Among these are:
there is only one set of source files to account for,
it allows one to standardize the build procedure,
and it provides a clearer starting point for any future ports.
In fact,
a build procedure is now part of the standard SUPES distribution and
is documented in Chapter~\ref{sec:install}.
Further,
the system dependent modules set an appropriate template for the
porting of SUPES to other hardware and/or software configurations.

Applications programmers face many similar user and system interface problems
during code development.  Because ANSI standard FORTRAN does not address many of
these problems, each programmer solves these problems for his/her own code.
SUPES aids the programmer by: 
\begin{enumerate} 

\item Providing a library of useful subprograms.

\item Defining a standard interface format for common utilities.

\item Providing a single point for debugging of common utilities.  That
is, SUPES has to be debugged only once and then is ready for use
by any code.
\end{enumerate}

Use of SUPES by the applications programmer can expand a code's capability,
reduce errors, minimize support effort and reduce development time.  Because
SUPES was designed to be reliable and supportable, there are some features
that are not included.  (1) It is not extremely sophisticated, rather it is
reliable and maintainable.  (2) Except for the extension library (Chapter
4), it is not system dependent.  (3) It does not take advantage of extended
system capabilities since they may not be available on a wide range of
operating systems.  (4) It is not written to maximize cpu speed.

It is the intention of the authors to maintain SUPES on all scientific
computer systems commonly used by Engineering Sciences Directorate (1500)
staff.
Currently these systems include:
\begin{enumerate}
\item Sun 3 and Sun 4 running SunOS operating system version 4.0.3 and later,

\item VAXen running VMS version 4.5 and later,

\item Cray X/MP and Y/MP running UNICOS version 5.0 and later, and

\item Alliant F/X 8 running Concentrix 5.0.0.
\end{enumerate}
A notable omission to the above list is the Cray running either CTSS
or the COS operating systems.
These configurations still require the FORTRAN source code for the extension
library that was provided in previous
implementations of SUPES~\cite{SUPES}.
This code continues to be included in the current standard SUPES distribution,
though a build procedure designed for these systems is not.
Specific ports of the SUPES utilities to new machines and/or operating systems will be
added to the original source files as the need arises.
Other Sandia personnel may obtain copies of SUPES from the
authors.  SUPES will also be available to non-Sandia personnel through the
National Energy Software Center.


    %-----------------------------%
    \chapter{Getting Started}\label{sec:getting started}
    This section is meant to get you using \muelu{} as quickly as possible.  \S\ref{sec:overview} gives a
summary of \muelu's design.  \S\ref{sec:configuration and build} lists \muelu's dependencies on other
\trilinos libraries and provides a sample cmake configuration line.  Finally, code examples using the XML
interface are given in \S\ref{sec:examples in code}.

\label{sec:overview}
\section{Overview of \muelu}
%algorithm types
%problems types
\muelu{} is an extensible algebraic multigrid (AMG) library that is part of the
\trilinos{} project. \muelu{} works with \epetra (32-bit version
\footnote{Support for the Epetra 64-bit version is planned.}) and
\tpetra matrix types. The library is written in C++ and allows for different
ordinal (index) and scalar types.  \muelu{} is designed to be efficient on many
different computer architectures, from workstations to supercomputers, relying
on ``MPI+X" principle, where ``X" can be threading or CUDA.

\muelu{} provides a number of different multigrid algorithms:
\be
  \item smoothed aggregation AMG (for Poisson-like and elasticity problems);
  \item Petrov-Galerkin aggregation AMG (for convection-diffusion problems);
  \item energy-minimizing AMG;
  \item aggregation-based AMG for problems arising from the eddy current
    formulation of Maxwell's equations.
\ee
\muelu's software design allows for the rapid introduction of new multigrid algorithms.
The most important features of \muelu{} can be summarized as:
\begin{description}
  \item \textbf{Easy-to-use interface}

    \muelu{} has a user-friendly parameter input deck which covers
    most important use cases.  Reasonable defaults are provided for common problem types
    (see Table \ref{t:problem_types}).

  \item \textbf{Modern object-oriented software architecture}

    \muelu{} is written completely in C++ as a modular object-oriented multigrid
    framework, which provides flexibility to combine and reuse existing
    components to develop novel multigrid methods.

  \item \textbf{Extensibility}

    Due to its flexible design, \muelu{} is an excellent toolkit for
    research on novel multigrid concepts. Experienced multigrid users have full
    access to the underlying framework through an advanced XML based interface.
    Expert users may use and extend the C++ API directly.

  \item \textbf{Integration with \trilinos{} library}

    As a package of \trilinos, \muelu{} is well integrated into the \trilinos
    environment. \muelu{} can be used with either the \tpetra{} or \epetra{}
    (32-bit) linear algebra stack. It is templated on the local index, global
    index, scalar, and compute node types. This makes \muelu{} ready for
    future developments.

  \item \textbf{Broad range of supported platforms}

    \muelu{} runs on wide variety of architectures, from desktop workstations to
    parallel Linux clusters and supercomputers (~\cite{lin2014}).

  \item \textbf{Open source}

    \muelu{} is freely available through a simplified BSD license (see Appendix~\ref{sec:license}).
\end{description}

\section{Configuration and Build}
\label{sec:configuration and build}

\muelu{} has been compiled successfully under Linux with the following C++
compilers: GNU versions 4.1 and later, Intel versions 12.1/13.1, and clang versions 3.2 and later.
In the future, we recommend using compilers supporting C++11 standard.

\subsection{Dependencies}

\noindent{\bf Required Dependencies}

\muelu{} requires that \teuchos{} and either \epetra/\ifpack or \tpetra/\ifpacktwo
are enabled.

\noindent{\bf Recommended Dependencies}

We strongly recommend that you enable the following \trilinos libraries along with \muelu:

\begin{itemize}
  \item \epetra stack: \aztecoo, \epetra, \amesos, \ifpack, \isorropia, \galeri,
    \zoltan;
  \item \tpetra stack: \amesostwo, \belos, \galeri, \ifpacktwo, \tpetra,
    \zoltantwo.
\end{itemize}

\noindent{\bf Tutorial Dependencies}

In order to run the \muelu{} Tutorial \cite{MueLuTutorial} located in \verb!muelu/doc/Tutorial!, \muelu{} must be configured with the following
dependencies enabled:

  \aztecoo, \amesos, \amesostwo, \belos, \epetra, \ifpack, \ifpacktwo, \isorropia,
  \galeri, \tpetra, \zoltan, \zoltantwo.

\begin{mycomment}
Note that the \muelu{} tutorial \cite{MueLuTutorial} comes with a VirtualBox image with a pre-installed
Linux and \trilinos{}.   In this way, a user can immediately begin experimenting with \muelu{} without
having to install the \trilinos{} libraries. Therefore, it is an ideal starting point to get in touch with \muelu{}.
\end{mycomment}

\noindent{\bf Complete List of Direct Dependencies}

\begin{table}[ht]
  \centering
  \begin{tabular}{p{3.5cm} c c c c}
    \toprule
    \multirow{2}{*}{Dependency} & \multicolumn{2}{c}{Required} & \multicolumn{2}{c}{Optional} \\
    \cmidrule(r){2-3} \cmidrule(l){4-5}
                   & Library  & Testing  & Library  & Testing        \\
    \hline
    \amesos        &          &          & $\times$ & $\times$  \\
    \amesostwo     &          &          & $\times$ & $\times$  \\
    \aztecoo       &          &          &          & $\times$  \\
    \belos         &          &          &          & $\times$  \\
    \epetra        &          &          & $\times$ & $\times$  \\
    \ifpack        &          &          & $\times$ & $\times$  \\
    \ifpacktwo     &          &          & $\times$ & $\times$  \\
    \isorropia     &          &          & $\times$ & $\times$  \\
    \galeri        &          &          &          & $\times$  \\
    \kokkosclassic &          &          & $\times$ & \\
    \teuchos{}     & $\times$ & $\times$ &          & \\
    \tpetra        &          &          & $\times$ & $\times$  \\
    \xpetra        & $\times$ & $\times$ &          & \\
    \zoltan        &          &          & $\times$ & $\times$  \\
    \zoltantwo     &          &          & $\times$ & $\times$  \\
    \midrule
    Boost          &          &          & $\times$ & \\
    BLAS           & $\times$ & $\times$ &          & \\
    LAPACK         & $\times$ & $\times$ &          & \\
    MPI            &          &          & $\times$ & $\times$  \\
    \bottomrule
  \end{tabular}
  \caption{\label{tab:dependencies}\muelu's required and optional dependencies,
    subdivided by whether a dependency is that of the \muelu{} library itself
    (\textit{Library}), or of some \muelu{} test (\textit{Testing}). }
\end{table}

Table~\ref{tab:dependencies} lists the dependencies of \muelu, both required and
optional. If an optional dependency is not present, the tests requiring that
dependency are not built.

\begin{mycomment}
\amesos{}/\amesostwo{} are necessary if one wants to use a sparse direct solve on the coarsest level.
\zoltan{}/\zoltantwo{} are necessary if one wants to use matrix rebalancing in parallel runs (see~\S\ref{sec:performance}).
\aztecoo{}/\belos{} are necessary if one wants to test \muelu{} as a preconditioner instead of a solver.
\end{mycomment}

\begin{mycomment}
\muelu{} has also been successfully tested with SuperLU 4.1 and SuperLU 4.2.
\end{mycomment}
\begin{mycomment}
Some packages that \muelu{} depends on may come with additional requirements for
third party libraries, which are not listed here as explicit dependencies of \muelu{}.
It is highly recommended to install ParMetis 3.1.1 or newer for \zoltan{},
ParMetis 4.0.x for \zoltantwo{}, and SuperLU 4.1 or newer for
\amesos{}/\amesostwo{}.
\end{mycomment}

\subsection{Configuration}
The preferred way to configure and build \muelu{} is to do that outside of the source directory.
Here we provide a sample configure script that will enable \muelu{} and all of its optional dependencies:
\begin{lstlisting}
  export TRILINOS_HOME=/path/to/your/Trilinos/source/directory
  cmake \
      -D BUILD_SHARED_LIBS:BOOL=ON \
      -D CMAKE_BUILD_TYPE:STRING="RELEASE" \
      -D CMAKE_CXX_FLAGS:STRING="-g" \
      -D Trilinos_ENABLE_EXPLICIT_INSTANTIATION:BOOL=ON \
      -D Trilinos_ENABLE_TESTS:BOOL=OFF \
      -D Trilinos_ENABLE_EXAMPLES:BOOL=OFF \
      -D Trilinos_ENABLE_MueLu:BOOL=ON \
      -D MueLu_ENABLE_TESTS:STRING=ON \
      -D MueLu_ENABLE_EXAMPLES:STRING=ON \
      -D TPL_ENABLE_BLAS:BOOL=ON \
      -D TPL_ENABLE_MPI:BOOL=ON \
  ${TRILINOS_HOME}
\end{lstlisting}

\noindent
More configure examples can be found in \texttt{Trilinos/sampleScripts}.
For more information on configuring, see the \trilinos Cmake Quickstart guide \cite{TrilinosCmakeQuickStart}.

\section{Examples in code}
\label{sec:examples in code}
% simple scaling test
%   galeri
%   XML input
%   belos/aztecoo or stand-alone solver
%   look @ tutorial or elsewhere for more advanced usage

The most commonly used scenario involving \muelu{} is using a multigrid
preconditioner inside an iterative linear solver. In \trilinos{}, a user has a
choice between \epetra and \tpetra for the underlying linear algebra library.
Important Krylov subspace methods (such as preconditioned CG and GMRES) are
provided in \trilinos{} packages \aztecoo (\epetra{}) and \belos
(\epetra{}/\tpetra{}).

At this point, we assume that the reader is comfortable with \teuchos{} referenced-counted
pointers (RCPs) for memory management (an introduction to RCPs can be found
in~\cite{RCP2010}) and the \texttt{Teuchos::ParameterList} class~\cite{TeuchosURL}.

\subsection{\muelu{} as a preconditioner within \belos}
\label{sec:tpetraexample}
The following code shows the basic steps of how to use a \muelu{}
multigrid preconditioner with \tpetra{} linear algebra library and with a linear
solver from \belos{}. To keep the example short and clear, we skip the template
parameters and focus on the algorithmic outline of setting up
a linear solver. For further details, a user may refer to the \texttt{examples} and
\texttt{test} directories.

First, we create the \muelu{} multigrid preconditioner. It can be done in a few
ways. For instance, multigrid parameters can be read from an XML file
(e.g., \textit{mueluOptions.xml} in the example below).
\begin{lstlisting}[language=C++]
    Teuchos::RCP<Tpetra::CrsMatrix<> > A;
    // create A here ...
    std::string optionsFile = "mueluOptions.xml";
    Teuchos::RCP<MueLu::TpetraOperator> mueLuPreconditioner =
       MueLu::CreateTpetraPreconditioner(A, optionsFile);
\end{lstlisting}
The XML file contains multigrid options. A typical file with \muelu{} parameters
looks like the following.
\begin{lstlisting}[language=XML]
<ParameterList name="MueLu">

  <Parameter name="verbosity" type="string" value="low"/>

  <Parameter name="max levels" type="int" value="3"/>
  <Parameter name="coarse: max size" type="int" value="10"/>

  <Parameter name="multigrid algorithm" type="string" value="sa"/>

  <!-- Damped Jacobi smoothing -->
  <Parameter name="smoother: type" type="string" value="RELAXATION"/>
  <ParameterList name="smoother: params">
    <Parameter name="relaxation: type"  type="string" value="Jacobi"/>
    <Parameter name="relaxation: sweeps" type="int" value="1"/>
    <Parameter name="relaxation: damping factor" type="double" value="0.9"/>
  </ParameterList>

  <!-- Aggregation -->
  <Parameter name="aggregation: type" type="string" value="uncoupled"/>
  <Parameter name="aggregation: min agg size" type="int" value="3"/>
  <Parameter name="aggregation: max agg size" type="int" value="9"/>

</ParameterList>
\end{lstlisting}
It defines a three level smoothed aggregation multigrid algorithm. The
aggregation size is between three and nine(2D)/27(3D) nodes.  One sweep with a
damped Jacobi method is used as a level smoother. By default, a direct solver is
applied on the coarsest level. A complete list of available parameters and valid
parameter choices is given in \S\ref{sec:muelu_options} of this User's Guide.

Users can also construct a multigrid preconditioner using a provided \parameterlist
without accessing any files in the following manner.
\begin{lstlisting}[language=C++]
  Teuchos::RCP<Tpetra::CrsMatrix<> > A;
  // create A here ...
  Teuchos::ParameterList paramList;
  paramList.set("verbosity", "low");
  paramList.set("max levels", 3);
  paramList.set("coarse: max size", 10);
  paramList.set("multigrid algorithm", "sa");
  // ...
  Teuchos::RCP<MueLu::TpetraOperator> mueLuPreconditioner =
     MueLu::CreateTpetraPreconditioner(A, paramList);
\end{lstlisting}

Besides the linear operator $A$, we also need an initial guess vector for the
solution $X$ and a right hand side vector $B$ for solving a linear system.
\begin{lstlisting}[language=C++]
    Teuchos::RCP<const Tpetra::Map<> > map = A->getDomainMap();

    // Create initial vectors
    Teuchos::RCP<Tpetra::MultiVector<> > B, X;
    X = Teuchos::rcp( new Tpetra::MultiVector<>(map,numrhs) );
    Belos::MultiVecTraits<>::MvRandom( *X );
    B = Teuchos::rcp( new Tpetra::MultiVector<>(map,numrhs) );
    Belos::OperatorTraits<>::Apply( *A, *X, *B );
    Belos::MultiVecTraits<>::MvInit( *X, 0.0 );
\end{lstlisting}
To generate a dummy example, the above code first declares two vectors. Then, a
right hand side vector is calculated as the matrix-vector product of a random vector
with the operator $A$. Finally, an initial guess is initialized with zeros.

Then, one can define a \texttt{Belos::LinearProblem} object where the
\texttt{mueLuPreconditioner} is used for left preconditioning
\begin{lstlisting}[language=C++]
    Belos::LinearProblem<> problem( A, X, B );
    problem->setLeftPrec(mueLuPreconditioner);
    bool set = problem.setProblem();
\end{lstlisting}

Next, we set up a \belos{} solver using some basic parameters
\begin{lstlisting}[language=C++]
    Teuchos::ParameterList belosList;
    belosList.set( "Block Size", 1 );
    belosList.set( "Maximum Iterations", 100 );
    belosList.set( "Convergence Tolerance", 1e-10 );
    belosList.set( "Output Frequency", 1 );
    belosList.set( "Verbosity", Belos::TimingDetails + Belos::FinalSummary );

    Belos::BlockCGSolMgr<> solver( rcp(&problem,false), rcp(&belosList,false) );
\end{lstlisting}

Finally, we solve the system.
\begin{lstlisting}[language=C++]
    Belos::ReturnType ret = solver.solve();
\end{lstlisting}

\subsection{\muelu{} as a preconditioner for \aztecoo}

For \epetra, users have two library options: \belos{} (recommended) and \aztecoo{}.
\aztecoo{} and \belos both provide fast and mature implementations of common iterative Krylov linear solvers.
\belos has additional capabilities, such as Krylov subspace recycling and ``tall skinny QR".

Constructing a \muelu{} preconditioner for Epetra operators is done in a similar
manner to Tpetra.
\begin{lstlisting}[language=C++]
    Teuchos::RCP<Epetra_CrsMatrix> A;
    // create A here ...
    Teuchos::RCP<MueLu::EpetraOperator> mueLuPreconditioner;
    std::string optionsFile = "mueluOptions.xml";
    mueLuPreconditioner = MueLu::CreateEpetraPreconditioner(A, optionsFile);
\end{lstlisting}
\muelu{} parameters are generally Epetra/Tpetra agnostic, thus the XML parameter file
could be the same as~\S\ref{sec:tpetraexample}.

Furthermore, we assume that a right hand side vector and a solution vector with
the initial guess are defined.
\begin{lstlisting}[language=C++]
    Teuchos::RCP<const Epetra_Map> map = A->DomainMap();
    Teuchos::RCP<Epetra_Vector> B = Teuchos::rcp(new Epetra_Vector(map));
    Teuchos::RCP<Epetra_Vector> X = Teuchos::rcp(new Epetra_Vector(map));
    X->PutScalar(0.0);
\end{lstlisting}

Then, an \texttt{Epetra\_LinearProblem} can be defined.
\begin{lstlisting}[language=C++]
    Epetra_LinearProblem epetraProblem(A.get(), X.get(), B.get());
\end{lstlisting}

The following code constructs an \aztecoo{} CG solver.
\begin{lstlisting}[language=C++]
    AztecOO aztecSolver(epetraProblem);
    aztecSolver.SetAztecOption(AZ_solver, AZ_cg);
    aztecSolver.SetPrecOperator(mueLuPreconditioner.get());
\end{lstlisting}

Finally, the linear system is solved.
\begin{lstlisting}[language=C++]
    int maxIts = 100;
    double tol = 1e-10;
    aztecSolver.Iterate(maxIts, tol);
\end{lstlisting}

\subsection{Further remarks}

This section is only meant to give a brief introduction on how to use \muelu{}
as a preconditioner within the \trilinos{} packages for iterative solvers. There
are other, more complicated, ways to use \muelu{} as a preconditioner for \belos
and \aztecoo through the \xpetra interface. Of course, \muelu{} can also work as
standalone multigrid solver. For more information on these topics, the reader
may refer to the examples and tests in the \muelu{} source directory
(\texttt{Trilinos/packages/muelu}), as well as to the \muelu{} tutorial~\cite{MueLuTutorial}.
For in-depth details of \muelu applied to multiphysics problems, please see~\cite{Wiesner2014}.


    %-----------------------------%
    \chapter{\ifpacktwo options}
    \label{sec:muelu_options}

In this section, we report the complete list of \muelu{} input parameters.  It
is important to notice, however, that \muelu{} relies on other \trilinos{}
packages to provide support for some of its algorithms. For instance,
\ifpack{}/\ifpacktwo{} provide standard smoothers like Jacobi, Gauss-Seidel or
Chebyshev, while \amesos{}/\amesostwo{} provide access to direct solvers. The
parameters affecting the behavior of the algorithms in those packages are
simply passed by \muelu{} to a routine from the corresponding library. Please
consult corresponding packages' documentation for a full list of supported
algorithms and corresponding parameters.

\section{Using parameters on individual levels}
Some of the parameters that affect the preconditioner can in principle be
different from level to level. By default, parameters affect all levels in
a multigrid hierarchy.

The settings on a particular levels can be changed by using level sublists.
A level sublist is a \parameterlist{} sublist with the name ``level XX'', where XX is the level number. The
parameter names in the sublist do not require any modifications. For example,
the following fragment of code
\begin{lstlisting}[language=XML]
  <ParameterList name="level 2">
    <Parameter name="smoother: type" type="string" value="CHEBYSHEV"/>
  </ParameterList>
\end{lstlisting}
changes the smoother for level 2 to be a polynomial smoother.

\section{Parameter validation}
By default, \muelu{} validates the input parameter list. A parameter that is
misspelled, is unknown, or has an incorrect value type will cause an exception to be
thrown and execution to halt.

\begin{mycomment}
Spaces are important within a parameter's name. Please separate words
by just one space, and make sure there are no leading or trailing spaces.
\end{mycomment}

The option \verb|print initial parameters| prints the initial list given to the
interpreter. The option \verb|print unused parameters| prints the list of unused
parameters.

% ==================== GENERAL ====================
\section{General options}
\label{sec:options_general}

\begin{table}[h!]
  \begin{center}
    \begin{tabular}{p{3cm} p{12cm}}
      \toprule
      Verbosity level           & Description \\
      \midrule
      \verb!none!               & No output \\
      \verb!low!                & Errors, important warnings, and some statistics \\
      \verb!medium!             & Same as \verb!low!, but with more statistics \\
      \verb!high!               & Errors, all warnings, and all statistics \\
      \verb!extreme!            & Same as \verb!high!, but also includes output from other packages (\textit{i.e.}, \zoltan{}) \\
      \bottomrule
    \end{tabular}
    \caption{Verbosity levels.}
\label{t:verbosity_types}
  \end{center}
\end{table}

\begin{table}[h!]
  \begin{center}
    \begin{tabular}{p{4.3cm} p{4.3cm} c p{4.5cm}}
      \toprule
      Problem type               & Multigrid algorithm    & Block size  & Smoother \\
      \midrule
      \verb!unknown!             & --                     & --          & -- \\
      \verb!Poisson-2D!          & Smoothed aggregation   & 1           & Chebyshev \\
      \verb!Poisson-3D!          & Smoothed aggregation   & 1           & Chebyshev \\
      \verb!Elasticity-2D!       & Smoothed aggregation   & 2           & Chebyshev \\
      \verb!Elasticity-3D!       & Smoothed aggregation   & 3           & Chebyshev \\
      \verb!ConvectionDiffusion! & Petrov-Galerkin  AMG   & 1           & Gauss-Seidel \\
      \verb!MHD!                 & Unsmoothed aggregation & --          & Additive Schwarz method with one level of overlap and ILU(0) as a subdomain solver \\
      \bottomrule
    \end{tabular}
    \caption{Supported problem types (``--'' means not set).}
\label{t:problem_types}
  \end{center}
\end{table}


\cbb{problem: type}{string}{"unknown"}{Type of problem to be solved. Possible values: see Table~\ref{t:problem_types}.}
          
\cbb{verbosity}{string}{"high"}{Control of the amount of printed information. Possible values: see Table~\ref{t:verbosity_types}.}
          
\cbb{number of equations}{int}{1}{Number of PDE equations at each grid node. Only constant block size is considered.}
          
\cbb{max levels}{int}{10}{Maximum number of levels in a hierarchy.}
          
\cbb{cycle type}{string}{"V"}{Multigrid cycle type. Possible values: "V", "W".}
          
\cbb{problem: symmetric}{bool}{true}{Symmetry of a problem. This setting affects the construction of a restrictor. If set to true, the restrictor is set to be the transpose of a prolongator. If set to false, underlying multigrid algorithm makes the decision.}
          
\cbb{xml parameter file}{string}{""}{An XML file from which to read additional
      parameters.  In case of a conflict, parameters manually set on
      the list will override parameters in the file. If the string is
      empty a file will not be read.}
          

% ==================== SMOOTHERS ====================
\section{Smoothing and coarse solver options}
\label{sec:options_smoothing}

\muelu{} relies on other \trilinos{} packages to provide level smoothers and
coarse solvers. \ifpack{} and \ifpacktwo{} provide standard smoothers (see
Table~\ref{tab:smoothers}), and \amesos{} and \amesostwo{} provide direct
solvers (see Table~\ref{tab:coarse_solvers}). Note that it is completely possible to use
any level smoother as a direct solver.

\muelu{} checks parameters \verb|smoother: * type| and \verb|coarse: type| to
determine:
\begin{itemize}
  \item what package to use (i.e., is it a smoother or a direct solver);
  \item (possibly) transform a smoother type

    \ding{42} \ifpack{} and \ifpacktwo{} use different smoother type names,
    e.g., ``point relaxation stand-alone'' vs ``RELAXATION''.  \muelu{} tries to follow
    \ifpacktwo{} notation for smoother types. Please consult \ifpacktwo{}
    documentation~\cite{Ifpack2} for more information.
\end{itemize}
The parameter lists \verb|smoother: * params| and \verb|coarse: params| are
passed directly to the corresponding package without any examination of their
content. Please consult the documentation of the corresponding packages for a list of
possible values.

By default, \muelu{} uses one sweep of symmetric Gauss-Seidel for both pre- and
post-smoothing, and SuperLU for coarse system solver.

\begin{table}[tbh]
  \begin{center}
    \begin{tabular}{p{4.0cm} p{10cm}}
      \toprule
      \texttt{smoother: type}           & \\
      \midrule
      \verb|RELAXATION|                 & Point relaxation smoothers, including
                                          Jacobi, Gauss-Seidel, symmetric Gauss-Seidel, etc. The exact
                                          smoother is chosen by specifying \texttt{relaxation: type} parameter in
                                          the \texttt{smoother: params} sublist. \\
      \verb|CHEBYSHEV|                  & Chebyshev polynomial smoother. \\
      \verb|ILUT|, \verb|RILUK|         & Local (processor-based) incomplete factorization methods. \\
      \bottomrule
    \end{tabular}
    \caption{Commonly used smoothers provided by \ifpack{}/\ifpacktwo{}. Note
    that these smoothers can also be used as coarse grid solvers.}
\label{tab:smoothers}
  \end{center}
\end{table}

\begin{table}[tbh]
  \begin{center}
    \begin{tabular}{p{4.0cm} c c p{7cm}}
      \toprule
      \texttt{coarse: type}             & \amesos{} & \amesostwo{} &  \\
      \midrule
      \verb|KLU|                        & x & & Default \amesos{} solver~\cite{klu}. \\
      \verb|KLU2|                       & & x & Default \amesostwo{} solver~\cite{amesos2_belos}. \\
      \verb|SuperLU|                    & x & x & Third-party serial sparse direct solver~\cite{Li2011}. \\
      \verb|SuperLU_dist|               & x & x & Third-party parallel sparse direct solver~\cite{Li2011}. \\
      \verb|Umfpack|                    & x & & Third-party solver~\cite{umfpack}. \\
      \verb|Mumps|                      & x & & Third-party solver~\cite{mumps}. \\
      \bottomrule
    \end{tabular}
    \caption{Commonly used direct solvers provided by \amesos{}/\amesostwo{}}
\label{tab:coarse_solvers}
  \end{center}
\end{table}

In certain cases, the user may want to do no smoothing on a particular level, or do no solve on the coarsest level.
\begin{itemize}
  \item To skip smoothing, use the option \verb!smoother: pre or post! with value \verb!none!.
  \item To skip the coarse grid solve, use the option \verb!coarse: type! with value \verb!none!.
\end{itemize}


\cbb{smoother: pre or post}{string}{"both"}{Pre- and post-smoother combination. Possible values: "pre" (only pre-smoother), "post" (only post-smoother), "both" (both pre-and post-smoothers), "none" (no smoothing).}
          
\cbb{smoother: type}{string}{"RELAXATION"}{Smoother type. Possible values: see Table~\ref{tab:smoothers}.}
          
\cbb{smoother: pre type}{string}{"RELAXATION"}{Pre-smoother type. Possible values: see Table~\ref{tab:smoothers}.}
          
\cbb{smoother: post type}{string}{"RELAXATION"}{Post-smoother type. Possible values: see Table~\ref{tab:smoothers}.}
          
\cba{smoother: params}{\parameterlist}{Smoother parameters. For standard smoothers, \muelu passes them directly to the appropriate package library.}
          
\cba{smoother: pre params}{\parameterlist}{Pre-smoother parameters. For standard smoothers, \muelu passes them directly to the appropriate package library.}
          
\cba{smoother: post params}{\parameterlist}{Post-smoother parameters. For standard smoothers, \muelu passes them directly to the appropriate package library.}
          
\cbb{smoother: overlap}{int}{0}{Smoother subdomain overlap.}
          
\cbb{smoother: pre overlap}{int}{0}{Pre-smoother subdomain overlap.}
          
\cbb{smoother: post overlap}{int}{0}{Post-smoother subdomain overlap.}
          
\cbb{coarse: max size}{int}{2000}{Maximum dimension of a coarse grid. \muelu will stop coarsening once it is achieved.}
          
\cbb{coarse: type}{string}{"SuperLU"}{Coarse solver. Possible values: see Table~\ref{tab:coarse_solvers}.}
          
\cba{coarse: params}{\parameterlist}{Coarse solver parameters. \muelu passes them directly to the appropriate package library.}
          
\cbb{coarse: overlap}{int}{0}{Coarse solver subdomain overlap.}
          

% ==================== AGGREGATION ====================
\section{Aggregation options}
\label{sec:options_aggregation}

\begin{table}[h!]
  \begin{center}
    \begin{tabular}{p{5.0cm} p{10cm}}
      \toprule
      \verb!uncoupled! & Attempts to construct aggregates of optimal size ($3^d$
                         nodes in $d$ dimensions). Each process works independently, and
                         aggregates cannot span multiple processes.\\
      \verb!coupled!   & Attempts to construct aggregates of optimal size ($3^d$
                         nodes in $d$ dimensions). Aggregates are allowed to
                         cross processor boundaries. \textbf{Use carefully}. If
                         unsure, use \verb!uncoupled! instead.\\
      \verb!brick!     & Attempts to construct rectangular aggregates \\
      %\verb!METIS!     & Use graph partitioning algorithm to create aggregates,
      %                   working process-wise. Number of nodes in each aggregate
      %                   is specified with option \texttt{aggregation: max agg
      %                   size}. \\
      % \verb!ParMETIS!  & As \verb!METIS!, but partition global graph. Aggregates
                         % can span arbitrary number of processes. Specify global
                         % number of aggregates with {\tt aggregation: global
                         % number}. \\
      \bottomrule
    \end{tabular}
    \caption{Available coarsening schemes. }
\label{t:aggregation}
  \end{center}
\end{table}


\cbb{aggregation: type}{string}{"uncoupled"}{Aggregation scheme. Possible values: see Table~\ref{t:aggregation}.}
          
\cbb{aggregation: ordering}{string}{"natural"}{Node ordering strategy. Possible values: "natural" (local index order), "graph" (filtered graph breadth-first order), "random" (random local index order).}
          
\cbb{aggregation: drop scheme}{string}{"classical"}{Connectivity dropping scheme for a graph used in aggregation. Possible values: "classical", "distance laplacian".}
          
\cbb{aggregation: drop tol}{double}{0.0}{Connectivity dropping threshold for a graph used in aggregation.}
          
\cbb{aggregation: min agg size}{int}{2}{Minimum size of an aggregate.}
          
\cbb{aggregation: max agg size}{int}{-1}{Maximum size of an aggregate (-1 means unlimited).}
          
\cbb{aggregation: brick x size}{int}{2}{Number of points for x axis in "brick" aggregation (limited to 3).}
          
\cbb{aggregation: brick y size}{int}{2}{Number of points for y axis in "brick" aggregation (limited to 3).}
          
\cbb{aggregation: brick z size}{int}{2}{Number of points for z axis in "brick" aggregation (limited to 3).}
          
\cbb{aggregation: Dirichlet threshold}{double}{0.0}{Threshold for determining whether entries are zero during Dirichlet row detection.}
          
\cbb{aggregation: export visualization data}{bool}{false}{Export data for visualization post-processing.}
          
\cbb{aggregation: output filename}{string}{""}{Filename to write VTK visualization data to.}
          
\cbb{aggregation: output file: time step}{int}{0}{Time step ID for non-linear problems.}
          
\cbb{aggregation: output file: iter}{int}{0}{Iteration for non-linear problems.}
          
\cbb{aggregation: output file: agg style}{string}{Point Cloud}{Style of aggregate visualization.}
          
\cbb{aggregation: output file: fine graph edges}{bool}{false}{Whether to draw all fine node connections along with the aggregates.}
          
\cbb{aggregation: output file: coarse graph edges}{bool}{false}{Whether to draw all coarse node connections along with the aggregates.}
          
\cbb{aggregation: output file: build colormap}{bool}{false}{Whether to output a random colormap in a separate XML file.}
          

% ==================== REBALANCING ====================
\section{Rebalancing options}
\label{sec:options_rebalancing}


\cbb{repartition: enable}{bool}{false}{Rebalancing on/off switch.}
          
\cbb{repartition: partitioner}{string}{"zoltan2"}{Partitioning package to use. Possible values: "zoltan" (\zoltan{} library), "zoltan2" (\zoltantwo{} library).}
          
\cba{repartition: params}{\parameterlist}{Partitioner parameters. \muelu passes them directly to the appropriate package library.}
          
\cbb{repartition: start level}{int}{2}{Minimum level to run partitioner. \muelu does not rebalance levels finer than this one.}
          
\cbb{repartition: min rows per proc}{int}{800}{Minimum number of rows per processor. If the actual number if smaller, then rebalancing occurs.}
          
\cbb{repartition: max imbalance}{double}{1.2}{Maximum nonzero imbalance ratio. If the actual number is larger, the rebalancing occurs.}
          
\cbb{repartition: remap parts}{bool}{true}{Postprocessing for partitioning to reduce data migration.}
          
\cbb{repartition: rebalance P and R}{bool}{false}{Explicit rebalancing of R and P during the setup. This speeds up the solve, but slows down the setup phases.}
          

% ==================== MULTIGRID ====================
\section{Multigrid algorithm options}
\label{sec:options_mg}

\begin{table}[h!]
  \begin{center}
    \begin{tabular}{p{3.5cm} p{11cm}}
      \toprule
      \verb!sa!         & Classic smoothed aggregation~\cite{VMB1996} \\
      \verb!unsmoothed! & Aggregation-based, same as \verb!sa! but without damped Jacobi prolongator improvement step \\
      \verb!pg!         & Prolongator smoothing using $A$, restriction smoothing using $A^T$, local damping factors~\cite{ST2008} \\
      \verb!emin!       & Constrained minimization of energy in basis functions of grid transfer operator~\cite{WTWG2014,OST2011} \\
      \bottomrule
    \end{tabular}
    \caption{Available multigrid algorithms for generating grid transfer matrices. }
\label{t:mgs}
  \end{center}
\end{table}


\cbb{multigrid algorithm}{string}{"sa"}{Multigrid method. Possible values: see Table~\ref{t:mgs}.}
          
\cbb{sa: damping factor}{double}{1.33}{Damping factor for smoothed aggregation.}
          
\cbb{sa: use filtered matrix}{bool}{true}{Matrix to use for smoothing the tentative prolongator. The two options are: to use the original matrix, and to use the filtered matrix with filtering based on filtered graph used for aggregation.}
          
\cbb{filtered matrix: use lumping}{bool}{true}{Lump (add to diagonal) dropped entries during the construction of a filtered matrix. This allows user to preserve constant nullspace.}
          
\cbb{filtered matrix: reuse eigenvalue}{bool}{true}{Skip eigenvalue calculation during the construction of a filtered matrix by reusing eigenvalue estimate from the original matrix. This allows us to skip heavy computation, but may lead to poorer convergence.}
          
\cbb{emin: iterative method}{string}{"cg"}{Iterative method to use for energy minimization of initial prolongator in energy-minimization. Possible values: "cg" (conjugate gradient), "gmres" (generalized minimum residual), "sd" (steepest descent).}
          
\cbb{emin: num iterations}{int}{2}{Number of iterations to minimize initial prolongator energy in energy-minimization.}
          
\cbb{emin: num reuse iterations}{int}{1}{Number of iterations to minimize the reused prolongator energy in energy-minimization.}
          
\cbb{emin: pattern}{string}{"AkPtent"}{Sparsity pattern to use for energy minimization. Possible values: "AkPtent".}
          
\cbb{emin: pattern order}{int}{1}{Matrix order for the "AkPtent" pattern.}
          

% ==================== REUSE ====================
\section{Reuse options}
\label{sec:options_reuse}

Reuse options are a way for a user to speed up the setup stage of multigrid.
The main requirement to use reuse is that the matrix' graph structure does not
change. Only matrix values are allowed to change.

The reuse options control the degree to which multigrid hierarchy is preserved
for a subsequent setup call.

In addition, please note that not all combinations of multigrid algorithms and
reuse options are valid, or even make sense. For instance, the "emin" reuse
option should only be use with "emin" multigrid algorithm.

Table~\ref{t:reuse_types} contains the information about different reuse
options. The options are ordered in increasing number of reuse components, from
the no reuse to the full reuse ("full").

\begin{table}[h!]
  \begin{center}
    \begin{tabular}{p{3.0cm} p{12cm}}
      \toprule
      \verb!none!   & No reuse \\
      \verb!S!      & Reuse only the symbolic information of the level smoothers. \\
      \verb!tP!     & Reuse tentative prolongator. The graphs of smoothed
                      prolongator and matrices in Galerkin product are reused
                      only if filtering is not being used ({\it i.e.}, either
                      \verb!sa: use filtered matrix! or \verb!aggregation: drop tol! is false) \\
      \verb!emin!   & Reuse old prolongator as an initial guess to energy
                      minimization, and reuse the prolongator pattern \\
      \verb!RP!     & Reuse smoothed prolongator and restrictor. Smoothers are
                      recomputed.  \ding{42} \verb!RP! should reuse matrix graphs for
                      matrix-matrix product, but currently that is disabled as only \epetra{}
                      supports it. \\
      \verb!RAP!    & Recompute only the finest level smoothers, reuse all other operators \\
      \verb!full!   & Reuse everything \\
      \bottomrule
    \end{tabular}
    \caption{Available coarsening schemes. }
\label{t:reuse_types}
  \end{center}
\end{table}


\cbb{reuse: type}{string}{"none"}{Reuse options for consecutive hierarchy construction. This speeds up the setup phase, but may lead to poorer convergence. Possible values: see Table~\ref{t:reuse_types}.}
          

% ==================== MISCELLANEOUS ====================
\section{Miscellaneous options}


\cba{export data}{\parameterlist}{Exporting a subset of the hierarchy data in a
      file. Currently, the list can contain any of the following parameter
      names (``A'', ``P'', ``R'', ``Nullspace'', ``Coordinates'') of type \texttt{string}
      and value ``\{levels separated by commas\}''. A
      matrix/multivector with a name ``X'' is saved in two or three
      three MatrixMarket files: a) data is saved in
      \textit{X\_level.mm}; b) its row map is saved in
      \textit{rowmap\_X\_level.mm}; c) its column map (for matrices) is saved in
      \textit{colmap\_X\_level.mm}.}
          
\cbb{print initial parameters}{bool}{true}{Print parameters provided for a hierarchy construction.}
          
\cbb{print unused parameters}{bool}{true}{Print parameters unused during a hierarchy construction.}
          
\cbb{transpose: use implicit}{bool}{false}{Use implicit transpose for the restriction operator.}
          


    %-----------------------------%
    % \chapter{Performance}
    % \section{How to wring the last bit of performance out of Ifpack2 (jhu,csiefer)}
\section{Published results}
Cite the PPL paper \cite{Lin2014}.


    %\nocite{*}

    % ---------------------------------------------------------------------- %
    % References
    %
    \clearpage
    % If hyperref is included, then \phantomsection is already defined.
    % If not, we need to define it.
    \providecommand*{\phantomsection}{}
    \phantomsection
    \addcontentsline{toc}{chapter}{References}
    \bibliographystyle{plain}
    \bibliography{ifpack2guide}


    % ---------------------------------------------------------------------- %
    %
    \appendix
    \input{appendix}
    %\chapter{Historical Perspective}
	%\input{CommonHistory}


    %\chapter{Some Other Appendix}
	%\input{CommonAppendix}

    % \printindex

    %
% This is an example of how to create the distribution page. Some
% distributions are required by Sandia; e.g. the housekeeping copies.
% Depending on the type of report; e.g. CRADA, Patent Caution, etc.
% additional distribution lines may have to be added. See the
% "Guide for Preparing SAND Reports"
%
% SANDdistribution takes CA or NM as an optional argument. If given,
% the approrpiate housekeeping copies are inserted autmatically.
% Inside the SANDdistribution environment, several commands can be used
% insert the distributions for CRADA, LDRD, etc. See example below.
%
% You can leave the CA or NM option off and not use any of the SANDdist*
% commands. This will allow you to create a distribution list manually.
%
\begin{SANDdistribution}[NM]
    % Housekeeping copies necessary for every unclassified report:
    % \SANDdistCRADA	% If this report is about CRADA work
    % \SANDdistPatent	% If this report has a Patent Caution or Patent Interest
    % \SANDdistLDRD	% If this report is about LDRD work

    % Some external Addresses
    %\SANDdistExternal{1}{An Address\\ 99 $99^{th}$ street NW\\City, State}
    %\SANDdistExternal{3}{Some Address\\ and street\\City, State}
    %\SANDdistExternal{12}{Another Address\\ On a street\\City, State\\U.S.A.}
    \SANDdistExternal{1}{Tobias Wiesner\\Institute for Computational Mechanics \\Technische Universit\"at
    M\"unchen\\Boltzmanstra\ss e 15 \\85747 Garching, Germany}
    \bigskip


    % The following MUST BE between the external and internal distributions!
    % \SANDdistClassified % If this report is classified


    % Internal Addresses
    \SANDdistInternal{1}{1320}{Michael Heroux}{1446}
    \SANDdistInternal{1}{1318}{Robert Hoekstra}{1446}
    \SANDdistInternal{1}{1320}{Mark Hoemmen}{1446}
    \SANDdistInternal{1}{1320}{Paul Lin}{1446}
    \SANDdistInternal{1}{1318}{Andrey Prokopenko}{1426}
    \SANDdistInternal{1}{1322}{Christopher Siefert}{1443}

\end{SANDdistribution}

\begin{SANDdistribution}[CA]
    \SANDdistInternal{1}{9159}{Jonathan Hu}{1426}
    \SANDdistInternal{1}{9159}{Paul Tsuji}{1442}
    \SANDdistInternal{1}{9159}{Raymond Tuminaro}{1442}
\end{SANDdistribution}


\end{document}
